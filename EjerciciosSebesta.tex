%!TEX TS-program = pdflatex
%!TEX encoding = UTF-8 Unicode

\documentclass[11pt]{article}

\usepackage[utf8]{inputenc}
\usepackage{geometry}
\geometry{a4paper}
\usepackage{graphicx}
\usepackage{booktabs}
\usepackage{array}
\usepackage{alltt}
\usepackage{verbatim}
\usepackage{subfig}
\usepackage{hyperref}


\usepackage{url}

\usepackage{fancyhdr} 
\pagestyle{fancy}
\renewcommand{\headrulewidth}{0pt} 
\lhead{}\chead{}\rhead{}
\lfoot{}\cfoot{\thepage}\rfoot{}

\usepackage{sectsty}
\allsectionsfont{\sffamily\mdseries\upshape}

\usepackage[nottoc,notlof,notlot]{tocbibind} 
\usepackage[titles,subfigure]{tocloft} 
\renewcommand{\cftsecfont}{\rmfamily\mdseries\upshape}
\renewcommand{\cftsecpagefont}{\rmfamily\mdseries\upshape}

\usepackage[spanish]{babel}
\usepackage{listings}

%%%El documento comienza aqui

\title{\textbf{Ejercicios de Programming set del Libro Sebesta}}
\author{\textbf{Jimmy Banchon - Rene Balda}}
\date{\textbf{\today}}
\begin{document}
\maketitle

\section{Introducción}
\paragraph{} \noindent
Los capitulos a realizar son:
\begin{enumerate}
\item
 Ch5: 4 - 7
\item
 Ch6: 1, 2, 7
\item
 Ch7: 1 - 6, 9
\item
 Ch8: 3, 4, 5
\item
 Ch9: 1, 5
\end{enumerate}

\section{Preguntas y Respuestas}

\subsection{Capítulo 5: Nombres, Enlaces y Alcances.}
\subsubsection{Pregunta 4:}


\subsubsection{Pregunta 5:}
Para la prueba en C, se genera un error.
\begin{lstlisting}[frame=single]
void main(void)
{
	x=21;
	printf("%d",x);
	int x;
	x=42;
	printf("%d",x);
}
\end{lstlisting}

\noindent Errores:

\begin{verbatim}
	1.-	error C2065  x  identificador no declarado	
		c:\users\jimmy\documents\visual studio \2011\projects\test\test.c

	2.-	error C2065  x  identificador no declarado	
		c:\users\jimmy\documents\visual studio\2011\projects\test\test.c
\end{verbatim}

Para la prueba en C++, se dio este error al compilar.


\begin{lstlisting}[frame=single]
void main(void)
{
	
	cout << "Inserta 21:";
	cin >> x;
        int x;
	x=42;
	cout << "X vale:" << x;
	cin.get();cin.get();
	
}
\end{lstlisting}

\noindent Errores:

\begin{verbatim}
	1.-	error C2065: 'x' : identificador no declarado 
		c:\users\jimmy\documents\visual studio \2011\projects\test\test.cpp
\end{verbatim}

Para la prueba en java, se dio el siguiente error.

\begin{lstlisting}[frame=single]
public class Test {

    static void main(String[] args) {
        x=21;
        int x;
        x=42;
       System.out.println(x);
    }
}
\end{lstlisting}

\noindent Errores:

\begin{verbatim}
	Exception in thread "main" java.lang.RuntimeException: 
	Uncompilable source code - cannot find symbol
 	symbol:   variable x
	location: class test.NewMain
	at test.NewMain.main(NewMain.java:17)
Java Result: 1

\end{verbatim}



\subsubsection{Pregunta 6:}
Para la prueba en C++
\begin{lstlisting}[frame=single]

using namespace std;

int main(){
  for (int i = 0 ; i<5; i++){
    cout << i;
  }
  cout << i; 
  return 0;
}
\end{lstlisting}
\noindent Errores:
\begin{verbatim}
  Error test.cpp: En la función int main():
	test.cpp:8:11: error: la búsqueda de nombre de 'i'
			cambió por el nuevo alcance ISO de 'for'
			[-fpermissive]
\end{verbatim}

Para la prueba en Java
\begin{lstlisting}[frame=single]
public class Test{
    public static void main(String args[]){
	for(int i = 0; i< 5 ; i++)
	    {
		System.out.println(i);
	    }
	System.out.println(i);
    }
}
\end{lstlisting}
\noindent Errores:
\begin{verbatim}
Error	Test.java:9: error: cannot find symbol
	System.out.println(i);
	symbol:   variable i
	 location: class test
\end{verbatim}

Para la prueba en C sharp

\begin{lstlisting}[frame=single]
class c5p6
{
  static void Main()
  {
    for(int i = 0 ; i < 5 ; i++)
      {
	Console.WriteLine(i);
      }
    Console.WriteLine(i); 
  }

}
\end{lstlisting}
\noindent Errores:
\begin{verbatim}
Error	test.cs(11,23): error CS0103: The name `i' does 
	not exist in the current context
\end{verbatim}
Como podemos ver tanto en lenguaje C++, Java y C sharp no es posible hacer lo que nos pide el problema y nos genera un error, debido a que la variable solo puede ser accedida dentro del bloque for.

\subsubsection{Pregunta 7:}

\subsection{Capítulo 6: Tipos de Dato.}
\subsubsection{Pregunta 1:}

\subsubsection{Pregunta 2:}

\subsubsection{Pregunta 7:}

\subsection{Capítulo 7: Expressions and Assignment Statements}

\subsubsection{Pregunta 2:}
Traducción al lenguaje C++

\begin{lstlisting}[frame = single]

int fun(int *k); //Prototipo

int main() {
    int i = 10, j = 10, sum1, sum2;
    sum1 = (i / 2) + fun(&d);
    sum2 = fun(&j) + (j / 2);
    printf("sum1 =  %d  \n ",sum1);
    printf("sum2 =  %d ",sum2);
    return 0;
}

int fun(int *k) {
    *k += 4;
    return 3 * (*k) - 1;
}
\end{lstlisting}

\noindent Resultados:
\begin{verbatim}
	sum1 =  46
	sum2 =  48
\end{verbatim}

Dado los resultados obtenidos podemos concluir que la asociatividad es de izquierda a derecha al igual que C.

Traducción al lenguaje Java

\begin{lstlisting}[frame = single]
public class Test {
    public static void main(String[] args) {
        Test t = new Test();
        int i = 10, j = 10, sum1, sum2;
        sum1 = (i/2) + t.fun(i);
        sum2 = t.fun(j) + (j/2);
        System.out.println("Valor 1: "+sum1);
        System.out.println("Valor 2: "+sum2);
    }

    public int fun(int k){
        k += 4;
        return (3 * (k) - 1);
    }
}
\end{lstlisting}
\noindent Resultados:
\begin{verbatim}
	sum1 =  46
	sum2 =  46
\end{verbatim}

Traducción al lenguaje C sharp

\begin{lstlisting}[frame = single]
class Test
    {
        static void Main(string[] args)
        {
            Test t = new Program();
            int i = 10,j=10,sum1,sum2;
            sum1 = (i / 2) + t.fun(ref i);
            sum2 = t.fun(ref j) + (j / 2);
            System.Console.WriteLine(" "+sum1);
            System.Console.WriteLine(" "+sum2);
            Console.Read();
        }
        public int fun(ref int k)
        {
            k = 4 + k;
            return 3*(k) - 1;
        }
    }
\end{lstlisting}
\noindent Resultados:
\begin{verbatim}
	sum1 =  46
	sum2 =  48
\end{verbatim}
Dado los resultados obtenidos podemos concluir que la asociatividad es de izquierda a derecha al igual que C y C++.

\subsubsection{Pregunta 4:}

Lenguaje Java.

\begin{lstlisting}[frame=single] 

public class Test
    {
    final static int num=5;
    static int x=5;
   
    public static void main(String[] args){
        
          x = fun()+x; 
          System.out.println(x); 
    }

    static int fun() {
            x = 17;
            return 3;
    }

}
\end{lstlisting}

\noindent Resultados:
\begin{verbatim}
	x =  20
\end{verbatim}

Primeramente la funcion fun retorna 3 y actualiza la variable global x con 17 para luego sumarla con 3 asignandole 20 a la variable x.
En java el operador + tiene asociatividad de izquierda a derecha.

\subsubsection{Pregunta 5:}

Lenguaje C++

\begin{lstlisting}[frame=single] 

int fun();

extern int x = 10;
void main(){
	
	x = fun()+x;
	a = a+fun();
	printf("\%d", x);
	getch();
}

int fun() {
	x = 17;
	return 3;
}
\end{lstlisting}

\noindent Resultados:
\begin{verbatim}
	x =  20
\end{verbatim}

Para este lenguaje podemos ver que siempre en la función al llamar fun(), se va a actualizar la variable global a con 17 y luego se le sumara 3 teniendo asi un valor de 20 en dicha variable.


\subsubsection{Pregunta 6:}
Lenguaje C Sharp

\begin{lstlisting}[frame=single]
class Test
    {
        static int x = 5;
        static void Main(string[] args)
        {
            x = a +fun();
            Console.WriteLine(x);
            Console.ReadLine();
       }

        static int fun()
        {
            x = 17;
            return 3;
        }
    }
\end{lstlisting}

\noindent Resultados:
OUTPUT\\
\begin{verbatim}
	x =  8
\end{verbatim}

Como podemos ver en C Sharp también se cumple la regla de la asociativad.Es decir la asociatividad va de izquierda a derecha.


\subsection{Capítulo 8: Expressions and Assignment Statements}

\subsection{Capítulo 9: SubProgramas.}


        

\end{document}
