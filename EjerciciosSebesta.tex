%!TEX TS-program = pdflatex
%!TEX encoding = UTF-8 Unicode

\documentclass[11pt]{article}

\usepackage[utf8]{inputenc}
\usepackage{geometry}
\geometry{a4paper}
\usepackage{graphicx}
\usepackage{booktabs}
\usepackage{array}
\usepackage{verbatim}
\usepackage{subfig}
\usepackage{hyperref}

\usepackage{url}

\usepackage{fancyhdr} 
\pagestyle{fancy}
\renewcommand{\headrulewidth}{0pt} 
\lhead{}\chead{}\rhead{}
\lfoot{}\cfoot{\thepage}\rfoot{}

\usepackage{sectsty}
\allsectionsfont{\sffamily\mdseries\upshape}

\usepackage[nottoc,notlof,notlot]{tocbibind} 
\usepackage[titles,subfigure]{tocloft} 
\renewcommand{\cftsecfont}{\rmfamily\mdseries\upshape}
\renewcommand{\cftsecpagefont}{\rmfamily\mdseries\upshape}

\usepackage[spanish]{babel}
\usepackage{listings}

%%%El documento comienza aqui

\title{\textbf{Ejercicios de Programming set del Libro Sebesta}}
\author{\textbf{Jimmy Banchon - Rene Balda}}
\date{\textbf{\today}}
\begin{document}
\maketitle

\section{Introducción}
\paragraph{} \noindent
Los capitulos a realizar son:
\begin{enumerate}
\item
 Ch5: 4 - 7
\item
 Ch6: 1, 2, 7
\item
 Ch7: 1 - 6, 9
\item
 Ch8: 3, 4, 5
\item
 Ch9: 1, 5
\end{enumerate}

\section{Preguntas y Respuestas}

\subsection{Capítulo 5: Nombres, Enlaces y Alcances.}

\subsection{Capítulo 6: Tipos de Dato.}

\subsection{Capítulo 7: Expressions and Assignment Statements}

\subsection{Capítulo 8: Expressions and Assignment Statements}

\subsection{Capítulo 9: SubProgramas.}


        

\end{document}
